\documentclass[a4paper]{article}
%\usepackage[T1]{fontenc}
\usepackage[english]{babel}

\usepackage{amsmath}
\usepackage{amssymb,amsfonts,textcomp, graphicx}
\usepackage{graphics}

\usepackage{wrapfig}

\usepackage{parskip}

\usepackage{color}
\usepackage{array}
\usepackage{hhline}
\usepackage{subcaption}

\usepackage{textcomp}

\usepackage[hidelinks]{hyperref}

\setlength\tabcolsep{1mm}
\renewcommand\arraystretch{1.3}

\setlength\voffset{-1in}
\setlength\hoffset{-1in}
\setlength\topmargin{0.7874in}
\setlength\oddsidemargin{0.7874in}
\setlength\textheight{10.118099in}
\setlength\textwidth{6.6932993in}
\setlength\footskip{0.0cm}
\setlength\headheight{0cm}
\setlength\headsep{0cm}


\begin{document}

\newcommand\textstyleEmphasis[1]{\textit{#1}}
\renewcommand{\contentsname}{Table des mati\`eres}
\renewcommand\refname{R\'ef\'erences}

\renewcommand{\abstractname}{Pr\'eambule}
\title{\textbf{Projet Circuits Int\'egr\'es Radiofr\'equence \\ TP Adaptation en Puissance}}
\author{Mohamed Hage Hassan \\ Cl\'ement Cheung}
\date{29 Novembre 2017}
\maketitle
\thispagestyle{empty}


\iffalse
\clearpage
\fi

\tableofcontents
\clearpage

\iffalse

\begin{figure}[!htb]
\begin{center}
  \includegraphics[scale=0.47]{Echantillonneur-bloqueur.png}
  \caption{Sch\'ema d'un \'echantilloneur-bloqueur \`a capacit\'e commut\'ee}
\end{center}
\end{figure}

\fi

\section{Introduction}

\section{Imp\'edance et Admittance - Analyse sous Cadence}
\subsection{\'Etude th\'eorique}

On essaye en premier temps de retrouver le circuit \'equivalent au celui RC en s\'erie :
On a :
\begin{equation}
  Z_C = \frac{1}{jC\omega} \phantom{8} \phantom{8} \phantom{8} X_S = - \frac{1}{C\omega}j
\end{equation}

Sachant que :
\[
Q = \frac{\|X_S \|}{R} = \frac{1}{R_S C\omega} = 0.159 < 3
\]
\[
X_S = \frac{1}{C_S \omega} = 159.15
\]

On prend :
\begin{equation}
  \begin{split}
  R_p = R_S (1 + Q^2) \\
  X_p = X_S \frac{(1+Q^2)}{Q^2}
  \end{split}
\end{equation}

Ce qui nous donne $R_p = 1025,28 \Omega$, $X_p = 6.454 \times 10^3$
\[
X_P = \frac{1}{C_P \omega} \implies C_P = \frac{1}{X_P \omega} = 24.7 fF
\]

\subsection{Simulation sous Cadence}
\textcolor{red}{\textbf{Que repr\'esente $S_{11}$?}}\\
On effectue une simulation du circuit RC pour analyser les param\`etres $S_{11}$, $Z_{11}$ et $Y_{11}$

\begin{figure}[!htb]
  \begin{subfigure}[t]{.5\linewidth}
      \centering
      \includegraphics[width=1.1\linewidth]{circuit-RC.png}
      \label{fig:rccircuit}
  \end{subfigure}%
  \begin{subfigure}[t]{.5\linewidth}
    \centering
    \includegraphics[width=1.1\linewidth]{sim-inital.png}
    \label{fig:rccircuit-sim}
  \end{subfigure}%
  \caption{Sch\'ema et Simulation du circuit}
  \label{fig:RC-sim}
\end{figure}

\clearpage

On retrouve :
\begin{equation}
  \begin{split}
    S_{11} = 0.906911 - j 0.0141 \\
    Z_{d} = 20 - j 3.19066 \\
    Y_{d} = 0.04859 + j 0.00777
  \end{split}
\end{equation}

Sachant que $ Y = G + j B $
\[
G = \frac{1}{R_P} \implies R_p = \frac{1}{G} = 1025 \Omega
\]
\[
B = \frac{1}{X_P} \implies X_P = \frac{1}{B} = 6.442 \times 10^3
\]


\section{Adaptation \`a $Z_0$}

\subsection{Adaptation avec un transformateur d'imp\'edance}
\subsubsection{Annulation de la partie imaginaire}
On ajoute une inductance en s\'erie au circuit RC, pour annuler la partie imaginaire $X_S$.
\[
X_L = \omega L = - X_S \implies X_L = 159.15
\]
\[
L = \frac{X_L}{\omega} = 25.33 nH
\]

\begin{figure}[!htb]
  \begin{subfigure}[t]{.5\linewidth}
      \centering
      \includegraphics[width=1.1\linewidth]{circuit-LRC.png}
      \label{fig:rccircuit}
  \end{subfigure}%
  \begin{subfigure}[t]{.5\linewidth}
    \centering
    \includegraphics[width=1\linewidth]{sim-LRC.png}
    \label{fig:rccircuit-sim}
  \end{subfigure}%
  \caption{Sch\'ema et Simulation du circuit}
  \label{fig:RC-sim}
\end{figure}


\clearpage
\addcontentsline{toc}{section}{R\'ef\'erences}

\begin{thebibliography}{9}

\bibitem{sim-elec-cours}
\textit{RF Microelectronics,2nd edition}\\
\texttt{Behzad Razavi, Prentice Hall}

\end{thebibliography}


\end{document}
